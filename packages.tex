%Includes
%=============================

%\usepackage[latin1]{inputenc} 		%Caracteres francais
\usepackage[utf8]{inputenc} 		%Caracteres francais
\usepackage[T1]{fontenc} 		%Caracteres francais
\usepackage[francais]{babel}		%On ecrit en francais
\usepackage{graphicx}			%Pour les images
\usepackage[top=2.5cm, bottom=2.5cm, left=2cm, right=2cm]{geometry}%Marges etc...
\usepackage{fancybox}			%Des boites
\usepackage[tikz]{bclogo}		%Pour faire des boites avec logo (remarques, etc...)
\usepackage{xcolor}			%Pour d�finir les couleurs
\usepackage{titlesec} 			%Pour modifier l'aspet des titres (sections, etc...)
\usepackage{soul} 			%Pour souligner
\usepackage{ifthen}			%Package des conditions
\usepackage{fancyhdr}			%Pour l'entete et le pied de page
\usepackage{lastpage}			%Pour connaitre le nombre total de page pour le pied de page
\usepackage{subfig}			%Pour faire des figures dans les figures
\usepackage{amsmath}			%Pour faire des maths (environnement ``align'' notamment)
\usepackage{amsfonts}			%Pour faire des maths (ensemble des reels, notamment)
\usepackage{amssymb}
\usepackage{mathrsfs}			%Pour faire des maths (notamment le L de la transform�e de Laplace)
\usepackage{answers}
\usepackage{pifont}			%Pour faire des chiffres entour�s
\usepackage[colorlinks=true,linktocpage]{hyperref}	%Pour faire des liens
\usepackage{calc}			%Pour faire heightof et des calculs de coordonn�es
\usepackage{import}			%Permet de g�rer les chemins relatifs des fichiers dans Latex
\usepackage{multirow}			%Pour faire des tableau avec plusieurs lignes
\usepackage{tikz}			%Pour faire des dessins
\usetikzlibrary{calc}
\usepackage{pgfplots}			%Pour faire des dessins (bis) (voir --> http://bertrandmasson.free.fr/index.php?article28/comment-faire-de-beaux-graphiques-avec-tikz-et-pgfplots)
\usepackage{enumerate}			%Pour personnaliser les listes ``enumerate''
\usepackage{wasysym}			%Pour faire le symbole diametre
\usepackage{array}			%Tableaux (notamment alignements)
\usepackage{placeins} 	%Pour utiliser la commande \FloatBarrier pour vider la m�moire des figures.
\usepackage{calc} %Pour faire des calculs
\usepackage{eurosym} %package euro
\usepackage{epstopdf} % gestion du format eps
\usepackage{hyperref}%Gestion des liens hypertextes
\usepackage{fancyvrb}%Gestion des boites pour l'informatique
%\usepackage{listings}
\usepackage{listingsutf8}
\usepackage{fourier}
\lstset{
	language={Python},
	columns=flexible,
	basicstyle=\ttfamily,
	keywordstyle=\color{blue}, % je voudrais mettre les mots-clé en gras (\bfseries) mais c'est incompatible avec typewriter
	commentstyle=\color{gray}, % commentaires
	backgroundcolor=\color{green!10}, % couleur du fond
	frame=single, rulecolor=\color{green!10}, % encadrement + couleur de l'encadrement
	%numbers=left, numberstyle=\tiny, stepnumber=1, numbersep=5pt,
	showstringspaces=false, % supprime les espaces apparents dans les lignes de texte
	stringstyle=\color{red!50}\itshape,
	inputencoding=utf8,
	extendedchars=true,
   literate=
  {á}{{\'a}}1 {é}{{\'e}}1 {í}{{\'i}}1 {ó}{{\'o}}1 {ú}{{\'u}}1
  {Á}{{\'A}}1 {É}{{\'E}}1 {Í}{{\'I}}1 {Ó}{{\'O}}1 {Ú}{{\'U}}1
  {à}{{\`a}}1 {è}{{\`e}}1 {ì}{{\`i}}1 {ò}{{\`o}}1 {ù}{{\`u}}1
  {À}{{\`A}}1 {È}{{\'E}}1 {Ì}{{\`I}}1 {Ò}{{\`O}}1 {Ù}{{\`U}}1
  {ä}{{\"a}}1 {ë}{{\"e}}1 {ï}{{\"i}}1 {ö}{{\"o}}1 {ü}{{\"u}}1
  {Ä}{{\"A}}1 {Ë}{{\"E}}1 {Ï}{{\"I}}1 {Ö}{{\"O}}1 {Ü}{{\"U}}1
  {â}{{\^a}}1 {ê}{{\^e}}1 {î}{{\^i}}1 {ô}{{\^o}}1 {û}{{\^u}}1
  {Â}{{\^A}}1 {Ê}{{\^E}}1 {Î}{{\^I}}1 {Ô}{{\^O}}1 {Û}{{\^U}}1
  {œ}{{\oe}}1 {Œ}{{\OE}}1 {æ}{{\ae}}1 {Æ}{{\AE}}1 {ß}{{\ss}}1
  {ű}{{\H{u}}}1 {Ű}{{\H{U}}}1 {ő}{{\H{o}}}1 {Ő}{{\H{O}}}1
  {ç}{{\c c}}1 {Ç}{{\c C}}1 {ø}{{\o}}1 {å}{{\r a}}1 {Å}{{\r A}}1
  {€}{{\EUR}}1 {£}{{\pounds}}1
}
%\usepackage{minted}
\usepackage{multicol}
\usepackage{algorithm2e}
\usepackage{pgf, tikz} % dessins et courbes
%\usepackage{subcaption}
%\usepackage{tikz,pgfplots}
%\usepackage{tkz-tab} % tableaux de variations
%\usetikzlibrary{calc}
%%\usepackage{pgf}
%\usepackage{pgfplots}
%\usetikzlibrary{babel}
%\usepackage{circuitikz} % l est un "L" minuscule
\DeclareSymbolFontAlphabet{\mathcal}{symbols}
\usepackage{color}
\usepackage{colortbl}
\newcommand*\lstinputpath[1]{\lstset{inputpath=#1}}

\definecolor{gris25}{gray}{0.75}
\definecolor{bleu}{RGB}{50,50,220}
\definecolor{bleuf}{RGB}{42,94,171}
\definecolor{bleuc}{RGB}{231,239,247}
\definecolor{rougef}{RGB}{185,18,27}
\definecolor{rouge}{RGB}{255,40,40}
\definecolor{rougec}{RGB}{255,230,231}
\definecolor{vertf}{RGB}{103,126,82}
\definecolor{vert}{RGB}{40,150,40}
\definecolor{vertc}{RGB}{220,255,191}
\definecolor{violetf}{RGB}{112,48,160}
\definecolor{violetc}{RGB}{230,224,236}
\definecolor{jaunec}{RGB}{220,255,191}
\definecolor{jaune}{RGB}{255,246,100}
%\usepackage{rpcinematik}

%\usepackage[table]{xcolor}%Mettre des coleurs dans les tableaux

\usepackage{float}
\usepackage{bodegraph}

\usepackage{booktabs}
\usepackage{rotating}

\usepackage{amsmath}
\usepackage[overload]{empheq}

\usepackage{longtable,booktabs}

\usepackage{siunitx}

\usepackage{textcomp}

%\usepackage{stmaryrd}
\usepackage{pythontex}
\usepackage{supertabular}

\usepackage{esvect}

\usepackage[breakable]{tcolorbox}

\usepackage{standalone}
\usepackage{schemabloc}
\usepackage{cancel}

%%% Mise en forme des références croisées
\newcommand{\citedoc}[1]{\textbf{document \ref{#1}}}
\newcommand{\citequest}[1]{\textbf{question Q\ref{#1}}}
\newcommand{\citefig}[1]{\textbf{figure \ref{#1}}}
\newcommand{\citetab}[1]{\textbf{tableau~\ref{#1}}}
\newcommand{\citeannexe}[1]{\textbf{annexe \ref{#1}}}

%% Gestion des numéros de parties, paragraphes et questions
\newcounter{num_probleme} \setcounter{num_probleme}{0}
\newcounter{num_exercice} \setcounter{num_exercice}{0}
\newcounter{num_quest} \setcounter{num_quest}{0}
\newcounter{num_partie} \setcounter{num_partie}{0}
\newcounter{num_sspartie} \setcounter{num_sspartie}{0}
\newcounter{num_doc} \setcounter{num_doc}{0}
\newcounter{num_annexe} \setcounter{num_annexe}{0}


\newcommand{\probleme}[1]{\stepcounter{num_probleme}
			\setcounter{num_partie}{0}\vspace{2mm}
			{\begin{center} \textbf{\large PROBL\`{E}ME
			\arabic{num_probleme}\\ \vspace{2mm}#1}\end{center}
			\vspace{2mm}}}
%\newcommand{\titre}[1]{\begin{center} \textbf{\large #1}\end{center}
%			\vspace{2mm}}
%\newcommand{\exercice}[1]{\stepcounter{num_exercice}
%			\setcounter{num_partie}{0}\vspace{2mm}
%			{\begin{center} \textbf{\large EXERCICE
%			\arabic{num_exercice}\\ \vspace{2mm}#1}\end{center}
%			\vspace{2mm}}}
\newcommand{\paragraphe}[1]{\vspace{4mm} {\textbf{#1}
			\vspace{2mm}}}
\newcommand{\partie}[1]{\stepcounter{num_partie}\setcounter{num_sspartie}{0}
		   {\vspace{2mm}\begin{center} \textbf{\begin{large}Partie
		    \Roman{num_partie} - \end{large}
		    \begin{large}#1\end{large}}\\ \par \end{center}}}
\newcommand{\souspartie}[1]{\stepcounter{num_sspartie}{\vspace{2mm} {
		    \textbf{\Roman{num_partie}.\arabic{num_sspartie} - #1}
			\vspace{2mm}}}}



\usepackage{rp-sysml}
\usepackage{rpcinematik}
\usepackage{schemabloc}

\usetikzlibrary{calc}
\usepgflibrary{arrows}
\usetikzlibrary{trees}
\usetikzlibrary{plotmarks}

\usepackage{pdfpages}%inclure un pdf complet

\usepackage{xargs}


\usepackage{siunitx}

\makeatletter
%Includes
%=============================

\usepackage[latin1]{inputenc} 		%Caracteres francais
%\usepackage[utf8]{inputenc} 		%Caracteres francais
\usepackage[T1]{fontenc} 		%Caracteres francais
\usepackage[francais]{babel}		%On ecrit en francais
\usepackage{graphicx}			%Pour les images
\usepackage[top=2.5cm, bottom=2.5cm, left=2cm, right=2cm]{geometry}%Marges etc...
\usepackage{fancybox}			%Des boites
\usepackage[tikz]{bclogo}		%Pour faire des boites avec logo (remarques, etc...)
\usepackage{xcolor}			%Pour d�finir les couleurs
\usepackage{titlesec} 			%Pour modifier l'aspet des titres (sections, etc...)
\usepackage{soul} 			%Pour souligner
\usepackage{ifthen}			%Package des conditions
\usepackage{fancyhdr}			%Pour l'entete et le pied de page
\usepackage{lastpage}			%Pour connaitre le nombre total de page pour le pied de page
\usepackage{subfig}			%Pour faire des figures dans les figures
\usepackage{amsmath}			%Pour faire des maths (environnement ``align'' notamment)
\usepackage{amsfonts}			%Pour faire des maths (ensemble des reels, notamment)
\usepackage{amssymb}
\usepackage{mathrsfs}			%Pour faire des maths (notamment le L de la transform�e de Laplace)
\usepackage{answers}
\usepackage{pifont}			%Pour faire des chiffres entour�s
\usepackage[colorlinks=true,linktocpage]{hyperref}	%Pour faire des liens
\usepackage{calc}			%Pour faire heightof et des calculs de coordonn�es
\usepackage{import}			%Permet de g�rer les chemins relatifs des fichiers dans Latex
\usepackage{multirow}			%Pour faire des tableau avec plusieurs lignes
\usepackage{tikz}			%Pour faire des dessins
\usetikzlibrary{calc}
\usepackage{pgfplots}			%Pour faire des dessins (bis) (voir --> http://bertrandmasson.free.fr/index.php?article28/comment-faire-de-beaux-graphiques-avec-tikz-et-pgfplots)
\usepackage{enumerate}			%Pour personnaliser les listes ``enumerate''
\usepackage{wasysym}			%Pour faire le symbole diametre
\usepackage{array}			%Tableaux (notamment alignements)
\usepackage{placeins} 	%Pour utiliser la commande \FloatBarrier pour vider la m�moire des figures.
\usepackage{calc} %Pour faire des calculs
\usepackage{eurosym} %package euro
\usepackage{epstopdf} % gestion du format eps
\usepackage{hyperref}%Gestion des liens hypertextes
\usepackage{fancyvrb}%Gestion des boites pour l'informatique
\usepackage{listings}
\usepackage{fourier}
\lstset{
	language={Python},
	columns=flexible,
	basicstyle=\ttfamily,
	keywordstyle=\color{blue}, % je voudrais mettre les mots-clé en gras (\bfseries) mais c'est incompatible avec typewriter
	commentstyle=\color{gray}, % commentaires
	backgroundcolor=\color{green!10}, % couleur du fond
	frame=single, rulecolor=\color{green!10}, % encadrement + couleur de l'encadrement
	%numbers=left, numberstyle=\tiny, stepnumber=1, numbersep=5pt,
	showstringspaces=false, % supprime les espaces apparents dans les lignes de texte
	stringstyle=\color{red!50}\itshape,
}
\usepackage{multicol}
\usepackage{algorithm2e}
\usepackage{pgf, tikz} % dessins et courbes
\usepackage{tkz-tab} % tableaux de variations
\usepackage{circuitikz} % l est un "L" minuscule
\DeclareSymbolFontAlphabet{\mathcal}{symbols}
\usepackage{color}
\usepackage{colortbl}
\newcommand*\lstinputpath[1]{\lstset{inputpath=#1}}

\definecolor{gris25}{gray}{0.75}
\definecolor{bleu}{RGB}{50,50,220}
\definecolor{bleuf}{RGB}{42,94,171}
\definecolor{bleuc}{RGB}{231,239,247}
\definecolor{rougef}{RGB}{185,18,27}
\definecolor{rouge}{RGB}{255,40,40}
\definecolor{rougec}{RGB}{255,230,231}
\definecolor{vertf}{RGB}{103,126,82}
\definecolor{vert}{RGB}{40,150,40}
\definecolor{vertc}{RGB}{220,255,191}
\definecolor{violetf}{RGB}{112,48,160}
\definecolor{violetc}{RGB}{230,224,236}
\definecolor{jaunec}{RGB}{220,255,191}
\definecolor{jaune}{RGB}{255,246,100}
%\usepackage{rpcinematik}

%\usepackage[table]{xcolor}%Mettre des coleurs dans les tableaux

\usepackage{float}
\usepackage{bodegraph}

\usepackage{booktabs}
\usepackage{rotating}

\usepackage{amsmath}
\usepackage[overload]{empheq}

\usepackage{longtable,booktabs}

\usepackage{siunitx}

\usepackage{textcomp}


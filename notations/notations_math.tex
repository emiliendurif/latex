%%%%%%%%%%%%%%%%%%%%%%%%%%%
%
% Notations math�matiques
%
%%%%%%%%%%%%%%%%%%%%%%%%%%%%%%%

\newcommand{\ssi}{si et seulement si }
\newcommand{\indiceGauche}[2]	{\ensuremath{	{\vphantom{#2}}_{#1}{#2} 	}}
\newcommand{\exposantGauche}[2]	{\ensuremath{	{\vphantom{#2}}^{#1}{#2} 	}}
\newcommand{\transposee}[1]	{\ensuremath{	\exposantGauche{\mathit t}{#1}	}}%{\vphantom{#1}}_{\mathit t}{#1}	}}

% FONCTION
%%%%%%%%%%%%%%%%%%%%
\newcommand{\fonction}[2]	{\ensuremath{	{#1}_{\left(#2\right)}		}}
\newcommand{\f}[2]		{	\fonction{#1}{#2}		}
\newcommand{\deriv}[3][]	{\ifthenelse{\equal{#1}{}}	{\ensuremath{	\frac{d{#2}}{d{#3}}}}	{\ensuremath{	\left[\frac{d{#2}}{d{#3}}\right]_{#1}}}}
\newcommand{\atan}[1][]		{\ifthenelse{\equal{#1}{}}	{\ensuremath{	\tan^{-1}	}}	{\ensuremath{	\tan^{-1}\left(#1\right)	}}}


% ENSEMBLES
%%%%%%%%%%%%%%%%%%%%%%%%%%%%%%
\newcommand{\R}		{\ensuremath{\mathbb{R}}	}
\newcommand{\couple}[2]{\ensuremath{\left(#1,#2\right)}}	%Couple (ex : (U,V) )
\newcommand{\triplet}[3]{\ensuremath{\left(#1,#2,#3\right)}}	%Triplet (ex : (U,V,W) )
\newcommand{\quadruplet}[4]{\ensuremath{\left(#1,#2,#3,#4\right)}}	%Triplet (ex : (U,V,W) )



% GEOMETRIE
%%%%%%%%%%%%%%%%%%%%%%%%%%
\newcommand{\segment}[1]	{\ensuremath{	\left[#1\right]		}}		%Fait un sement
\newcommand{\droite}[1]		{\ensuremath{	\left(#1\right)		}}
\newcommand{\arc}[1]		{\ensuremath{	\overset{\frown}{#1}	}}		%Arc
\renewcommand{\angle}[1]	{\ensuremath{	\left(\widehat{#1}\right)	}} %(Red�fini : la fonction angle correspondait au symbole angle)
%%%%%%%%%%%%%%%%%%%%%%%%%%%%%%

%	Cinématique

%%%%%%%%%%%%%%%%%%%%%%%%%%%%%%%
\RequirePackage{tikz}



%Raccourcis
\newcommand{\CIR}	{centre instantané de rotation}
\newcommand{\cir}	{\CIR}
\newcommand{\Cir}	{Centre instantané de rotation}


%DDL
\newcommand{\Rx}	{\ensuremath{	R_x	}}
\newcommand{\Ry}	{\ensuremath{	R_y	}}
\newcommand{\Rz}	{\ensuremath{	R_z	}}

\newcommand{\Tx}	{\ensuremath{	T_x	}}
\newcommand{\Ty}	{\ensuremath{	T_y	}}
\newcommand{\Tz}	{\ensuremath{	T_z	}}



%Géométrie
\newcommand{\solide}[1]	{\ensuremath{	#1	}}		%Solide
\newcommand{\sS}[1]	{\solide{S_{#1}}}	%Solides S1, S2, ...
\newcommand{\s}[1]	{\ensuremath{(#1)}}	%Solides (1), (2), ...

%Repères
%\newcommand{\repere}[4]	{\ensuremath{	\left(#1,#2,#3,#4\right)	}}
\newcommand{\rR}[1]	{\ensuremath{	R_{#1}	}	}


%coordonnées variables fonction du temps
\newcommand{\xt}	{\fonction{x}{t}}
\newcommand{\yt}	{\fonction{y}{t}}
\newcommand{\zt}	{\fonction{z}{t}}
\newcommand{\rt}	{\fonction{r}{t}}
\newcommand{\thetat}	{\fonction{\theta}{t}}

\newcommand{\thetap}	{\ensuremath{\dot\theta}}

\newcommand{\xtp}	{\fonction{\dot{x}}{t}}
\newcommand{\ytp}	{\fonction{\dot{y}}{t}}
\newcommand{\ztp}	{\fonction{\dot{z}}{t}}
\newcommand{\rtp}	{\fonction{\dot{r}}{t}}
\newcommand{\thetatp}	{\fonction{\dot\theta}{t}}

%Vitesse
%\newcommand{\vVitesse}[3][]		{\ifthenelse{\equal{#1}{}}{\vecteur{V_{\left(#2/#3\right)}}}{\vecteur{V_{\left(#1\in#2/#3\right)}}}	}%{\vecteurChamp{V}{#2/#3}}{\vecteurChamp{V}{#1\in#2/#3}}}

\newcommand{\vVitesse}[3]		{\ifthenelse{\equal{#1}{00}}{\vecteur{V}_{\left(#2/#3\right)}}{\vecteur{V}{\left(#1\in#2/#3\right)}}	}%{\vecteurChamp{V}{#2/#3}}{\vecteurChamp{V}{#1\in#2/#3}}}

\newcommand{\vvVitessea}[3][]		{\ifthenelse{\equal{#1}{00}}{\vecteur{V}_{\left(#2/#3\right)}}{\vecteur{V}{\left(#1\in#2/#3\right)}}	}%{\vecteurChamp{V}{#2/#3}}{\vecteurChamp{V}{#1\in#2/#3}}}

\newcommand{\vAcceleration}[3][]	{\ifthenelse{\equal{#1}{}}{\vecteur{\Gamma_{#2/#3}}}{\vecteur{\Gamma_{#1\in#2/#3}}}}
\newcommand{\vRotation}[2]		{\vecteur{\Omega_{\left(#1/#2\right)}}}
\newcommand{\vPivotement}[2]		{\vecteur{{\Omega_p}_{\left(#1/#2\right)}}}
\newcommand{\vRoulement}[2]		{\vecteur{{\Omega_r}_{\left(#1/#2\right)}}}

%Déplacement de vitesse
\newcommand{\deplaceVitesse}[4]	{\ensuremath{	\vVitesse[#3]{#1}{#2}+\vecteur{#4#3}\wedge\vRotation{#1}{#2}	}}	%\deplaceVitesse{Solid}{Referentiel}{Pdepart}{Parrivee}


%petits déplacements
\newcommand{\vDeplacement}[3][]		{\ifthenelse{\equal{#1}{}}{\vecteur{U_{#2/#3}}}{\vecteur{U_{#1\in#2/#3}}}}
\newcommand{\vDep}[3][]			{\vDeplacement[#1]{#2}{#3}}
\newcommand{\vPetitDeplacement}[3][]	{\ifthenelse{\equal{#1}{}}{\vecteur{U_{#2/#3}}}{\vecteur{U_{#1\in#2/#3}}}}
\newcommand{\vPetitDep}[3][]		{\vPetitDeplacement[#1]{#2}{#3}}
\newcommand{\vPetiteRotation}[2]	{\vecteur{\theta_{#1/#2}}}
\newcommand{\vPetiteRot}[2]		{\vPetiteRotation{#1}{#2}}

\newcommand{\wx}			{\ensuremath{	\omega_x	}}
\newcommand{\wy}			{\ensuremath{	\omega_y	}}
\newcommand{\wz}			{\ensuremath{	\omega_z	}}

%Torseur
\newcommand{\V}				{\ensuremath{	\mathscr{V}	}}	%Symbole du torseur cinematique
\newcommand{\C}				{\ensuremath{	\mathscr{C}	}}	%Symbole du torseur cinetique
\newcommand{\Dyn}				{\ensuremath{	\mathscr{D}	}}	%Symbole du torseur dynamique
\newcommand{\tCinematique}[3][]		{\ensuremath{	\torseur{\V^{#1}_{\left(#2/#3\right)}}	}}%torseur cinematique
\newcommand{\tCinetique}[3][]		{\ensuremath{	\torseur{\C^{#1}_{\left(#2/#3\right)}}	}}%torseur cinetique
\newcommand{\tDynamique}[3][]		{\ensuremath{	\torseur{\Dyn^{#1}_{\left(#2/#3\right)}}	}}%torseur dynamique
\newcommand{\tV}[2]			{\tCinematique{#1}{#2}}	%torseur cinematique
\newcommand{\tC}[2]			{\tCinetique{#1}{#2}}	%torseur cinetique
\newcommand{\tDyn}[2]			{\tDynamique{#1}{#2}}	%torseur dynamique
\newcommand{\D}				{\ensuremath{	\mathscr{D}	}}	%Symbole du torseur petit déplacement
\newcommand{\tPetitDeplacement}[3][]	{\ensuremath{	\torseur{\D^{#1}_{\left(#2/#3\right)}}	}}%torseur petit déplacement
\newcommand{\tPetitDep}[3][]		{\tPetitDeplacement[#1]{#2}{#3}}
\newcommand{\tD}[3][]			{\tPetitDeplacement[#1]{#2}{#3}}

%Moments dynamiques et cinetiques
\newcommand{\vSigma}[3][]		{\ifthenelse{\equal{#1}{}}{\vecteur{\sigma_{\left(#2/#3\right)}}}{\vecteur{\sigma_{#1}\left(#2/#3\right)}}	}%{\vecteurChamp{V}{#2/#3}}{\vecteurChamp{V}{#1\in#2/#3}}}
\newcommand{\vDelta}[3][]		{\ifthenelse{\equal{#1}{}}{\vecteur{\delta_{\left(#2/#3\right)}}}{\vecteur{\delta_{#1}\left(#2/#3\right)}}	}%{\vecteurChamp{V}{#2/#3}}{\vecteurChamp{V}{#1\in#2/#3}}}
%Graphique / figures
%Figures planes de projection
%\newcommand{\scFigCalc}[6][]{
%\begin{tikzpicture}
%\draw[->] (0,0) -- (3,0) node[above] {$\overrightarrow{#2}$};
%\draw[->] (0,0) -- (0,3) node[right] {$\overrightarrow{#3}$};
%\draw[->] (0cm,0cm) -- (25:3cm) node[above] {$\overrightarrow{#4}$};
%\draw[->] (0cm,0cm) -- (115:3cm) node[right] {$\overrightarrow{#5}$};
%\draw[->] (1,0) arc (0:25:1cm)node[right=0.5em] {$#6$} ;
%\draw[->] (0,1) arc (90:115:1cm)node[above=0.5em] {$#6$} ;
%\node[below](0,0){$\overrightarrow{#1}$};
%\end{tikzpicture}
%}

%\newcommand*{\scFigCalc}{\@ifstar\scFigCalcSeul\scFigCalctikz}
%\newcommand{\scFigCalctikz}[6][]{\begin{tikzpicture}
%\begin{scope}[thick, >=latex]
%\draw[->] (0,0) -- (3,0) node[above] {$\vv{#2}$};
%\draw[->] (0,0) -- (0,3) node[right] {$\vv{#3}$};
%\draw[->] (0cm,0cm) -- (25:3cm) node[above] {$\vv{#4}$};
%\draw[->] (0cm,0cm) -- (115:3cm) node[right] {$\vv{#5}$};
%\draw[->] (1,0) arc (0:25:1cm)node[right=0.5em] {$#6$} ;
%\node[below](0,0){${#1}$};
%\end{scope}
%\end{tikzpicture}}

%\newcommand{\scFigCalcSeul}[6][]{
%\begin{scope}[thick, >=latex]
%\draw[->] (0,0) -- (3,0) node[above] {$\vv{#2}$};
%\draw[->] (0,0) -- (0,3) node[right] {$\vv{#3}$};
%\draw[->] (0cm,0cm) -- (25:3cm) node[above] {$\vv{#4}$};
%\draw[->] (0cm,0cm) -- (115:3cm) node[right] {$\vv{#5}$};
%\draw[->] (1,0) arc (0:25:1cm)node[right=0.5em] {$#6$} ;
%\node[below](0,0){${#1}$};
%\end{scope}}


%%% Matrices d'inertie

\newcommand{\inertie}[2]{I_{#1}\left( #2\right)}
\newcommand{\matinertie}[7]{
\begin{pmatrix}
#1 & #6 & #5 \\
#6 & #2 & #4 \\
#5 & #4 & #3 \\
\end{pmatrix}_{#7}}
%%%%%%%%%%%%%%

% Vecteur vitesse
 \newcommand{\vectv}[3]{%
\vect{V\left( {#1} \in {#2}/{#3}\right)}
}

% Vecteur force
\newcommand{\vectf}[2]{%
\vect{R\left( {#1} \rightarrow {#2}\right)}
}

% Vecteur moment stat
\newcommand{\vectm}[3]{%
\vect{\mathcal{M}\left( {#1}, {#2} \rightarrow {#3}\right)}
}


 \newcommand{\torseurcol}[7]{
\left\{%
\begin{array}{cc}%
{#1} & {#4}\\%
{#2} & {#5}\\%
{#3} & {#6}\\%
\end{array}%
\right\}_{#7}%
}

 \newcommand{\torseurl}[3]{%
%\left\{\mathcal{#1}\right\}_{#2}=%
\left\{%
\begin{array}{l}%
{#1} \\%
{#2} %
\end{array}%
\right\}_{#3}%
}

\newcommand{\torseurcin}[3]{%
\left\{\mathcal{#1} \left(#2/#3 \right) \right\}
}

\newcommand{\torseurci}[2]{%
\left\{\sigma \left(#1/#2 \right) \right\}
}
\newcommand{\torseurdyn}[2]{%
\left\{\delta \left(#1/#2 \right) \right\}
}

\newcommand{\torseurstat}[3]{%
\left\{\mathcal{#1} \left(#2\rightarrow #3 \right) \right\}
}

% d droit pour le calcul différentiel
\newcommand{\dd}{\text{d}}
\newcommand{\ddd}{\text{dd}}

% Vecteur accélération
 \newcommand{\vectg}[3]{%
\vect{\Gamma \left( {#1} \in {#2}/{#3}\right)}
}


% Vecteur résultante dyn
\newcommand{\vectrd}[2]{%
\vect{R_d \left( {#1}/ {#2}\right)}
}
% Vecteur moment dyn
\newcommand{\vectmd}[3]{%
\vect{\delta \left( {#1}, {#2} /{#3}\right)}
}


% Vecteur résultante cin
\newcommand{\vectrc}[2]{%
\vect{R_c \left( {#1}/ {#2}\right)}
}
% Vecteur moment cin
\newcommand{\vectmc}[3]{%
\vect{\sigma \left( {#1}, {#2} /{#3}\right)}
}



\newcommand{\axe}[2]{\couple{#1}{#2}}


%==============================================================================%
%	Graphes des liaisons
%==============================================================================%
\newenvironment{grapheLiaisons}[1][]{

\begin{tikzpicture}[#1]

\glConfig

}{\end{tikzpicture}}

\newcommandx{\glConfig}[2][1=,2=]{

\tikzstyle{liaison}=[align=center,#1,]

\tikzstyle{piece}=[circle,draw,align=center,#2,]

}

\newcommandx{\glPiece}[4][4=]{\node ({#2}) at ({#1})[piece,#4] {#3};}         % Pièce

\newcommandx{\glBati}[6][1=0,5=1,6=]{     % On a rajouté l'échelle et l'angle 

  \glPiece{#2}{#3}{#4}[#6]

  \gbati[#1]{#3}[#5][#6]

}             % Bâti  (le premier paramètre est l'angle)

\newcommandx{\glLiaison}[6][1=,2=,5=,6=above]{\draw [liaison,#2] ({#3}) to[#1] node[midway, #6]{#5} ({#4});}

\newcommandx{\glDeuxL}[3][1=c]{\begin{tabular}{#1} {#2}\\{#3} \end{tabular}}  % deprecated (au profit de [align=center] et \\)

\newcommandx{\gbati}[4][1=0,3=1,4=]{   % 1: angle, 2: position, 3: echelle, 4: parametres du noeud

\begin{scope}[rotate=#1,scale=#3]

\draw[#4] ({#2}.({#1}-90) --++(0,-0.2) coordinate (K);

\draw[#4] (K) --++(0.35,0);

\draw[#4] (K) --++(-0.35,0) coordinate (L);

\foreach \x in {0,0.14, ...,0.65} {\draw[#4] ($(L) + (\x,0)$) --++(0.13,-0.13);};

\end{scope}
}



%%%%%%%%%%%%%%%%%%%%%%%%%%%%%%

%	Cin�matique

%%%%%%%%%%%%%%%%%%%%%%%%%%%%%%%
\RequirePackage{tikz}



%Raccourcis
\newcommand{\CIR}	{centre instantan� de rotation}
\newcommand{\cir}	{\CIR}
\newcommand{\Cir}	{Centre instantan� de rotation}


%DDL
\newcommand{\Rx}	{\ensuremath{	R_x	}}
\newcommand{\Ry}	{\ensuremath{	R_y	}}
\newcommand{\Rz}	{\ensuremath{	R_z	}}

\newcommand{\Tx}	{\ensuremath{	T_x	}}
\newcommand{\Ty}	{\ensuremath{	T_y	}}
\newcommand{\Tz}	{\ensuremath{	T_z	}}



%G�om�trie
\newcommand{\solide}[1]	{\ensuremath{	#1	}}		%Solide
\newcommand{\sS}[1]	{\solide{S_{#1}}}	%Solides S1, S2, ...
\newcommand{\s}[1]	{\ensuremath{(#1)}}	%Solides (1), (2), ...

%Rep�res
\newcommand{\repere}[4]	{\ensuremath{	\left(#1,#2,#3,#4\right)	}}
\newcommand{\rR}[1]	{\ensuremath{	R_{#1}	}	}


%coordonn�es variables fonction du temps
\newcommand{\xt}	{\fonction{x}{t}}
\newcommand{\yt}	{\fonction{y}{t}}
\newcommand{\zt}	{\fonction{z}{t}}
\newcommand{\rt}	{\fonction{r}{t}}
\newcommand{\thetat}	{\fonction{\theta}{t}}

\newcommand{\thetap}	{\ensuremath{\dot\theta}}

\newcommand{\xtp}	{\fonction{\dot{x}}{t}}
\newcommand{\ytp}	{\fonction{\dot{y}}{t}}
\newcommand{\ztp}	{\fonction{\dot{z}}{t}}
\newcommand{\rtp}	{\fonction{\dot{r}}{t}}
\newcommand{\thetatp}	{\fonction{\dot\theta}{t}}

%Vitesse
\newcommand{\vVitesse}[3][]		{\ifthenelse{\equal{#1}{}}{\vecteur{V_{\left(#2/#3\right)}}}{\vecteur{V_{\left(#1\in#2/#3\right)}}}	}%{\vecteurChamp{V}{#2/#3}}{\vecteurChamp{V}{#1\in#2/#3}}}
\newcommand{\vAcceleration}[3][]	{\ifthenelse{\equal{#1}{}}{\vecteur{\Gamma_{#2/#3}}}{\vecteur{\Gamma_{#1\in#2/#3}}}}
\newcommand{\vRotation}[2]		{\vecteur{\Omega_{\left(#1/#2\right)}}}
\newcommand{\vPivotement}[2]		{\vecteur{{\Omega_p}_{\left(#1/#2\right)}}}
\newcommand{\vRoulement}[2]		{\vecteur{{\Omega_r}_{\left(#1/#2\right)}}}

%D�placement de vitesse
\newcommand{\deplaceVitesse}[4]	{\ensuremath{	\vVitesse[#3]{#1}{#2}+\vecteur{#4#3}\wedge\vRotation{#1}{#2}	}}	%\deplaceVitesse{Solid}{Referentiel}{Pdepart}{Parrivee}


%petits d�placements
\newcommand{\vDeplacement}[3][]		{\ifthenelse{\equal{#1}{}}{\vecteur{U_{#2/#3}}}{\vecteur{U_{#1\in#2/#3}}}}
\newcommand{\vDep}[3][]			{\vDeplacement[#1]{#2}{#3}}
\newcommand{\vPetitDeplacement}[3][]	{\ifthenelse{\equal{#1}{}}{\vecteur{U_{#2/#3}}}{\vecteur{U_{#1\in#2/#3}}}}
\newcommand{\vPetitDep}[3][]		{\vPetitDeplacement[#1]{#2}{#3}}
\newcommand{\vPetiteRotation}[2]	{\vecteur{\theta_{#1/#2}}}
\newcommand{\vPetiteRot}[2]		{\vPetiteRotation{#1}{#2}}

\newcommand{\wx}			{\ensuremath{	\omega_x	}}
\newcommand{\wy}			{\ensuremath{	\omega_y	}}
\newcommand{\wz}			{\ensuremath{	\omega_z	}}

%Torseur
\newcommand{\V}				{\ensuremath{	\mathscr{V}	}}	%Symbole du torseur cinematique
\newcommand{\C}				{\ensuremath{	\mathscr{C}	}}	%Symbole du torseur cinetique
\newcommand{\Dyn}				{\ensuremath{	\mathscr{D}	}}	%Symbole du torseur dynamique
\newcommand{\tCinematique}[3][]		{\ensuremath{	\torseur{\V^{#1}_{\left(#2/#3\right)}}	}}%torseur cinematique
\newcommand{\tCinetique}[3][]		{\ensuremath{	\torseur{\C^{#1}_{\left(#2/#3\right)}}	}}%torseur cinetique
\newcommand{\tDynamique}[3][]		{\ensuremath{	\torseur{\Dyn^{#1}_{\left(#2/#3\right)}}	}}%torseur dynamique
\newcommand{\tV}[2]			{\tCinematique{#1}{#2}}	%torseur cinematique
\newcommand{\tC}[2]			{\tCinetique{#1}{#2}}	%torseur cinetique
\newcommand{\tDyn}[2]			{\tDynamique{#1}{#2}}	%torseur dynamique
\newcommand{\D}				{\ensuremath{	\mathscr{D}	}}	%Symbole du torseur petit d�placement
\newcommand{\tPetitDeplacement}[3][]	{\ensuremath{	\torseur{\D^{#1}_{\left(#2/#3\right)}}	}}%torseur petit d�placement
\newcommand{\tPetitDep}[3][]		{\tPetitDeplacement[#1]{#2}{#3}}
\newcommand{\tD}[3][]			{\tPetitDeplacement[#1]{#2}{#3}}

%Moments dynamiques et cinetiques
\newcommand{\vSigma}[3][]		{\ifthenelse{\equal{#1}{}}{\vecteur{\sigma_{\left(#2/#3\right)}}}{\vecteur{\sigma_{#1}\left(#2/#3\right)}}	}%{\vecteurChamp{V}{#2/#3}}{\vecteurChamp{V}{#1\in#2/#3}}}
\newcommand{\vDelta}[3][]		{\ifthenelse{\equal{#1}{}}{\vecteur{\delta_{\left(#2/#3\right)}}}{\vecteur{\delta_{#1}\left(#2/#3\right)}}	}%{\vecteurChamp{V}{#2/#3}}{\vecteurChamp{V}{#1\in#2/#3}}}
%Graphique / figures
%Figures planes de projection
\newcommand{\scFigCalc}[6][]{
\begin{tikzpicture}
\draw[->] (0,0) -- (3,0) node[above] {$\overrightarrow{#2}$};
\draw[->] (0,0) -- (0,3) node[right] {$\overrightarrow{#3}$};
\draw[->] (0cm,0cm) -- (25:3cm) node[above] {$\overrightarrow{#4}$};
\draw[->] (0cm,0cm) -- (115:3cm) node[right] {$\overrightarrow{#5}$};
\draw[->] (1,0) arc (0:25:1cm)node[right=0.5em] {$#6$} ;
\draw[->] (0,1) arc (90:115:1cm)node[above=0.5em] {$#6$} ;
\node[below](0,0){$\overrightarrow{#1}$};
\end{tikzpicture}
}



%%%%%%%%%%%%%%%%%%%%%%%%%%%%%%%%%%%%%%%%%%%%
%
%	Notations vecteurs/torseurs
%
%%%%%%%%%%%%%%%%%%%%%%%%%%%%%%%%%%%%%%%%%%%%



%Commandes de base
%-----------------------------
\newcommand{\vect}[1]{\overrightarrow{#1}}
\newcommand{\vecteur}[1]	{\ensuremath{\overrightarrow{#1}}}	%Fait un vecteur
\newcommand{\vecteurIndice}[2]	{\ifthenelse{\equal{#2}{\ }}{\vecteur{{#1}}}{\vecteur{{#1}_{#2}}}}	%Fait un vecteur avec un indice (x1, y1...), ou un vecteur simple si indice = espace
\newcommand{\vecteurChamp}[2]	{\fonction{\vecteur{#1}}{#2}}
\newcommand{\bipoint}[2]	{\ensuremath{	\vecteur{\segment{#1#2}}	}}	%bipoint
\newcommand{\vLie}[2]		{\ensuremath{	\couple{#1}{#2}	}}			%Vecteur lie
\newcommand{\vGlissant}[2]	{\ensuremath{	\couple{#1}{#2}	}}			%Vecteur glissant
\newcommand{\fN}[2][] {\ensuremath{	\protect\overrightarrow{N_{#1\rightarrow#2}}}}			%Vecteur lie
\newcommand{\fT}[2][]	{\ensuremath{	\protect\overrightarrow{T_{#1\rightarrow#2}}}}
\newcommand{\fpbis}	{\ensuremath{	\protect\overrightarrow{f_{p (S_1\rightarrow S_2)}}}}
\newcommand{\angl}[2]{\left(\vect{#1},\vect{#2}\right)}

%Espaces
%--------------------------------
\newcommand{\eAffine}[1][3]	{\ensuremath{	\mathscr{E}^{#1}	}}	%Espace affine
\newcommand{\eVectoriel}[1][3]	{\ensuremath{	E^{#1}			}}	%Espace vectoriel



%Représentation des vecteurs
\newcommand{\vColonne}[2][]	{\ensuremath{	\left( \begin{array}{c} #2 \end{array} \right)_{#1}	}}	%Vecteur colonne (avec coordonnées)


%Operateurs Vectoriel

\newcommand{\norme}[1]		{\ensuremath{	\left\Vert #1 \right\Vert	}}	%Norme
\newcommand{\abs}[1]		{\ensuremath{	\left\vert #1 \right\vert	}}	%Valeur absolue
\newcommand{\prodMixte}[3]	{\ensuremath{	\left(#1\wedge#2\right)\cdot#3	}}	%produit mixte
\newcommand{\doubleProdVect}[3]	{\ensuremath{	#1\wedge\left(#2\wedge#3\right)	}}	%produit vectoriel



%VECTEUR PRE-FAVRIQUES
%------------------------------


\newcommand{\vNul}		{\ensuremath{	\overrightarrow{0}	}}	%Symbole d'une base
\newcommand{\vPreFab}[2][]	{\ifthenelse{\equal{#1}{}}	{\vecteur{#2}}	{\vecteurChamp{#2}{#1}}	}	%Vecteur qui choisit tout seul si c'est un vecteur simple ou un champ (présence d'un parametre ou non)


%Tout ce qui est vecteurs e_i
\newcommand{\ve}[1]{\vecteurIndice{e}{#1}}

\newcommand{\vex}{\ve{x}}	%e_x
\newcommand{\vey}{\ve{y}}	%e_y
\newcommand{\vez}{\ve{z}}	%e_z

%\newcommand{\e}{\ensuremath{\overrightarrow {e_2}}}
%\newcommand{\e}{\ensuremath{\overrightarrow {e_3}}}

\newcommand{\vxi}[1]{\vecteurIndice{x}{#1}}	%x_i
\newcommand{\vyi}[1]{\vecteurIndice{y}{#1}}	%y_i
\newcommand{\vzi}[1]{\vecteurIndice{z}{#1}}	%z_i

\newcommand{\vx}[1]{\vecteurIndice{x}{#1}}
\newcommand{\vy}[1]{\vecteurIndice{y}{#1}}	%y_i
\newcommand{\vz}[1]{\vecteurIndice{z}{#1}}	%z_i

%\newcommand{\vXzero}{\ensuremath{\vecteur{x_1}}}
%\newcommand{\vYun}{\ensuremath{\vecteur{y_1}}}
%\newcommand{\vZun}{\ensuremath{\vecteur{z_1}}}

%\newcommand{\vXun}{\ensuremath{\vecteur{x_1}}}
%\newcommand{\vYun}{\ensuremath{\vecteur{y_1}}}
%\newcommand{\vZun}{\ensuremath{\vecteur{z_1}}}

%\newcommand{\vXdeux}{\ensuremath{\vecteur{x_2}}}
%\newcommand{\vYdeux}{\ensuremath{\vecteur{y_2}}}
%\newcommand{\vZdeux}{\ensuremath{\vecteur{z_2}}}


\newcommand{\vn}[1][]	{	\vPreFab[#1]{n}	}

\newcommand{\vu}[1][]	{	\vPreFab[#1]{u}	}
\newcommand{\vU}[1][]	{	\vPreFab[#1]{U}	}
%\ifthenelse{\equal{#1}{}}	{\vecteur{U}}	{\vecteurChamp{U}{#1}}	}
%\newcommand{\vU}[1][]	{\ifthenelse{\equal{#1}{}}	{ \ensuremath{\overrightarrow{U}}  }{   \ensuremath{\overrightarrow{U_{(#1)}}}     }}	%Vecteur U ou champ U(M)

\newcommand{\ux}{\ensuremath{u_x}}
\newcommand{\uy}{\ensuremath{u_y}}
\newcommand{\uz}{\ensuremath{u_z}}

%\newcommand{\vv}[1][]	{	\vPreFab[#1]{v}	}
\newcommand{\vV}[1][]	{	\vPreFab[#1]{V}	}
\renewcommand{\vV}[1]{\vecteur{V_{#1}}}
\renewcommand{\vV}[1][]{\ifthenelse{\equal{#1}{}}{ \ensuremath{\overrightarrow{V}}  }{   \ensuremath{\overrightarrow{V_{(#1)}}}     }}
\newcommand{\Vx}{\ensuremath{v_x}}	\renewcommand{\vx}{\Vx}
\newcommand{\Vy}{\ensuremath{v_y}}	\renewcommand{\vy}{\Vy}
\newcommand{\Vz}{\ensuremath{v_z}}	\renewcommand{\vz}{\Vz}


\newcommand{\vw}[1][]	{	\vPreFab[#1]{w}	}
\newcommand{\vW}[1][]	{	\vPreFab[#1]{W}	}
%\ifthenelse{\equal{#1}{}}	{\vecteur{W}}	{\vecteurChamp{W}{#1}}	}
%\newcommand{\vW}[1][]{\ifthenelse{\equal{#1}{}}{ \ensuremath{\overrightarrow{W}}  }{   \ensuremath{\overrightarrow{W_{(#1)}}}     }}
\newcommand{\Wx}{\ensuremath{w_x}}	%\newcommand{\wx}{\Wx}
\newcommand{\Wy}{\ensuremath{w_y}}	%\newcommand{\wy}{\Wy}
\newcommand{\Wz}{\ensuremath{w_z}}	%\newcommand{\wz}{\Wz}

\newcommand{\vOM}[1][]	{	\vPreFab[#1]{OM}	}
\newcommand{\OM}[1][]	{	\vOM[#1]	}
\newcommand{\Mx}{\ensuremath{m_x}}	\newcommand{\mx}{\Mx}
\newcommand{\My}{\ensuremath{m_y}}	\newcommand{\my}{\My}
\newcommand{\Mz}{\ensuremath{m_z}}	\newcommand{\mz}{\Mz}

\newcommand{\vUun}{\ensuremath{\vecteur{U_1}}}
\newcommand{\vUn}{\ensuremath{\vecteur{U_n}}}
\newcommand{\vVun}{\ensuremath{\vecteur{U_1}}}
\newcommand{\vVp}{\ensuremath{\vecteur{V_p}}}


\newcommand{\vOP}[1][]	{	\vPreFab[#1]{OP}	}
\newcommand{\OP}[1][]	{	\vOP[#1]	}

\newcommand{\vAB}[1][]	{	\vPreFab[#1]{AB}	}
\newcommand{\AB}[1][]	{	\vAB[#1]	}

\newcommand{\vBA}[1][]	{	\vPreFab[#1]{BA}	}
\newcommand{\BA}[1][]	{	\vBA[#1]	}

\newcommand{\vOA}[1][]	{	\vPreFab[#1]{OA}	}
\newcommand{\OA}[1][]	{	\vOA[#1]	}

\newcommand{\vOB}[1][]	{	\vPreFab[#1]{OB}	}
\newcommand{\OB}[1][]	{	\vOB[#1]	}

\newcommand{\vi}[1]	{\ensuremath{	\vecteur{i_{#1}}	}}
\newcommand{\vj}[1]	{\ensuremath{	\vecteur{j_{#1}}	}}
\newcommand{\vk}[1]	{\ensuremath{	\vecteur{k_{#1}}	}}


%Bases
%-------------------------
\newcommand{\B}		{\ensuremath{	\mathscr{B}	}}	%Symbole d'une base
\newcommand{\Bxyz}	{\triplet{\vex}{\vey}{\vez}}
\newcommand{\Buvw}	{\triplet{\vu}{\overrightarrow{v}}{\vw}}
\newcommand{\base}[3]	{\ensuremath{\left(#1,#2,#3\right)}}	


%Repere
%-------------------------
\newcommand{\repere}[4]	{\ensuremath{	\left(#1,#2,#3,#4\right)	}}
\newcommand{\rep}[1]{\mathcal{R}_{#1}}

% Vecteur omega
 \newcommand{\vecto}[2]{%
\vect{\Omega\left( {#1}/{#2}\right)}
}




%Mise à zéro des variables
%==================================


%\newcommand	{\partie}		{Sciences de l'ingénieur}
%\newcommand	{\titre}		{Analyse fonctionnelle}
%\newcommand	{\numero}		{1}
%\newcommand	{\auteur}		{Emilien DURIF}
%\newcommand	{\etablissement}	{Lycée Gustave Eiffel de Dijon}
%\newcommand	{\discipline}		{Sciences de L'ingénieur}
%\newcommand	{\classe}		{Classe préparatoire P.T.S.I.}
%\newcommand	{\annee}		{2011 - 2012}
%\newcommand	{\icone}		{../../latex/images/logo_eiffel.png}
%\newcommand	{\competences}		{}

%\renewcommand	{\partie}		{Statique des solides}
\newcommand	{\cycle}		{C1 : Performances statiques et cinématiques des systèmes composés de chaine de solides}
\newcommand	{\cycleresume}		{C2 : Chaine de solides}
\newcommand	{\titreresume}	{}
\newcommand	{\titre}		{Modélisation des actions mécaniques et résolution d'un problème de statique}
\newcommand	{\numero}		{C2-1}
\newcommand	{\numerotd}		{1}
\newcommand	{\auteur}		{Emilien DURIF}
\newcommand	{\etablissement}	{Lycée La Martinière Monplaisir Reims}
\newcommand	{\discipline}		{Sciences Industrielles pour l'Ingénieur}
\newcommand	{\classe}		{Classe préparatoire P.S.I.}
\newcommand	{\annee}		{2016 - 2017}
\newcommand	{\icone}		{../../../../latex/images/logo_martiniere.jpg}
\newcommand	{\competences}		{}



%Mise en page des titres de parties
%======================================

%Sections
\renewcommand\thesection{\Roman{section}}%Numérotation en chiffres romains
\titleformat{\section}[hang]{\LARGE\bfseries}{\thesection.}{1em}{}[\rule{\linewidth}{.5pt}]
\newcommand{\sectionbreak}{\clearpage}%Change de page a chaque section
%\titlespacing*{\section}{0px}{20cm}{\wordsep}

%Subsection
\renewcommand\thesubsection{\arabic{subsection}}%Numérotation en chiffres arabes
\titlespacing*{\subsection}{0cm}{1cm}{0.5cm}

%Subsubsection
\renewcommand\thesubsubsection{\alph{subsubsection})}%Numérotation en lettres
\titlespacing*{\subsubsection}{1cm}{0.5cm}{0.3cm}





%Entete / Pied de page
%===========================
\pagestyle{fancy}	%On veut utiliser les entetes/pieds
\rhead{TP\numerotd\;\titreresume}		%Entete gauche --> Titre du document
\lhead{\textsc{\titre}}	%Entete droite --> Titre de la sections

\renewcommand{\footrulewidth}{1px}
\lfoot{\etablissement}	%Pied de gauche --> Etablissement
\cfoot{\thepage\ / \pageref{LastPage}}	%Pied Centre --> n°de page
\rfoot{\classe \\ Année \annee}











%Booléen de texte à trou
\newboolean{texteATrou}
\setboolean{texteATrou}{false}	%Petite condition qui choisit entre 2 formats d'image
\newcommand{\setTexteATrouOn}{\setboolean{texteATrou}{true}}	%Active le texte à trous
\newcommand{\setTexteATrouOff}{\setboolean{texteATrou}{false}}	%Désactive le texte à trous
\definecolor{couleurTrou}{rgb}{0,0.5,0}
\newcommand{\trou}[1]{\ifthenelse{\boolean{texteATrou}}{\ifthenelse{\boolean{makingOf}}{\textcolor{couleurTrou}{#1}}{\hphantom{#1}}}{#1}}

%Booleen d'affichage vectoriel ou bitmap
\newboolean{imageEnVectoriel}
\setboolean{imageEnVectoriel}{true}	%Petite condition qui choisit entre 2 formats d'image
\newboolean{corrige}
\setboolean{corrige}{true}	%Petite condition qui choisit entre 2 formats d'image
%Booleen d'affichage des corrections de TD
\newboolean{tdcorrige}
\setboolean{tdcorrige}{false}




\newboolean{correction}
\setboolean{correction}{false}
\newcommand{\correction}[2][]{\ifthenelse{\boolean{correction}}{#2}{#1}}

%\newcommand{\makingOf}[1]{\ifthenelse{\boolean{makingOf}}{}{#1}}









%Mode DISPLAY STYLE dans les tableau
%%%% debut macro %%%%
\newenvironment{disarray}%
 {\everymath{\displaystyle\everymath{}}\array}%
 {\endarray}
%%%% fin macro %%%%

%Pour ne pas apparaitre dans le menu$  $
%exemple :
% \tocless\subsection{blabla}
\newcommand{\nocontentsline}[3]{}
\newcommand{\tocless}[2]{\bgroup\let\addcontentsline=\nocontentsline#1{#2}\egroup}


%ITEMIZE

\AtBeginDocument{%
	\renewcommand{\labelitemi}	{\textbullet}
	\renewcommand{\labelitemii}	{$\circ$}%
	\renewcommand{\labelitemiii}	{>}%
	\renewcommand{\labelitemiv}	{-}%
}

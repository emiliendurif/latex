%%%%%%%%%%%%%%%%%%%%%%%%%%%%%%%%%%%%%%%%%%%%%%%%
% E.Pinault-Bigeard - s2i@pinault-bigeard.com
% http://s2i.pinault-bigeard.com
% CC BY-NC-SA 2.0 FR - http://creativecommons.org/licenses/by-nc-sa/2.0/fr/
%%%%%%%%%%%%%%%%%%%%%%%%%%%%%%%%%%%%%%%%%%%%%%%%
\documentclass[10pt]{article}

%%%%%%%%%%%%%%%%%%%%%%%%%%%%%%%%%%%%%%%%%%%%%%%%
% Package UPSTI_Document
%%%%%%%%%%%%%%%%%%%%%%%%%%%%%%%%%%%%%%%%%%%%%%%% 
\RequirePackage[QCM,noTypographie]{UPSTI_Document}             
\RequirePackage{multicol}

%---------------------------------%
% Paramètres du package
%---------------------------------%

% Variante
\newcommand{\UPSTIvariante}{2}

% Classe
% 1: PTSI				6: PSI*			11: TSI2		16: Spé
% 2: PT	(par défaut)	7: MPSI			12: ATS
% 3: PT*				8: MP			13: PC
% 4: PCSI				9: MP*			14: PC*
% 5: PSI				10: TSI1		15: Sup
\newcommand{\UPSTIidClasse}{2}

% Type de document
% 1: Cours (par défaut)		8: Fiche Synthèse    		15: Programme de colle 
% 2: TD     				9: Formulaire				16: QCM
% 3: TP						10: Memo
% 4: Colle					11: Dossier Technique
% 5: DS						12: Dossier Ressource
% 6: DM						13: Concours Blanc
% 7: Fiche Méthode			14: Document Réponses
\newcommand{\UPSTIidTypeDocument}{16} % QCM

% Titre dans l'en-tête
\newcommand{\UPSTItitreEnTete}{Cinématique}      

% Durée de l'activité (pour DS, DM et TP)
\newcommand{\UPSTIduree}{15 min} 

% Numéro (ajoute " n°1" après DS ou DM)
\newcommand{\UPSTInumero}{1}

% Version du document
\newcommand{\UPSTInumeroVersion}{1.0}

%----------------------------------------------- 
\UPSTIcompileVars		% "Compile" les variables
%%%%%%%%%%%%%%%%%%%%%%%%%%%%%%%%%%%%%%%%%%%%%%%% 

%---------------------------------%
% Barême
%---------------------------------%
% Barême pour les questions simples
% 	e = incohérent
% 	v = vide
% 	b = bonne
% 	m = mauvaise
\baremeDefautS{e=0,v=0,b=2,m=-0.5}

% Barême pour les questions multiples
% 	e = incohérent
% 	v = vide
% 	b = bonne
% 	m = mauvaise
% 	formula = Une formule "maison" que j'ai faite. Elle se lit comme ceci : 
\baremeDefautM{e=0,v=0,formula=(NBC+NMC>0 ? -0.25*NMC+1*NBC : ( NB>0 ? 0 : 0 ))}

%%%%%%%%%%%%%%%%%%%%%%%%%%%%%%%%%%%%%%%%%%%%%%%% 
% Début du document
%%%%%%%%%%%%%%%%%%%%%%%%%%%%%%%%%%%%%%%%%%%%%%%% 
\begin{document}


%---------------------------------%
% Groupes de question
%---------------------------------%

\element{figure1}{
%*************************%
\begin{question}{projection1}

\noindent
\begin{minipage}{\linewidth}
\begin{minipage}[c]{.25\linewidth}
\setCouleursParametrage{black}{UPSTIcustomColor1}
\parametrageAngulaire{\theta}{\vy1}{\vz1}{\vx1}{\vy2}{\vz2}[\vx2]
\end{minipage}\hfill
\begin{minipage}[c]{0.73\linewidth}
Projeter \vz1 dans \bB2 :
\begin{reponses}
	\bonne{$\cos\theta.\vz2+\sin\theta.\vy2$}
	\mauvaise{$-\cos\theta.\vz2+\sin\theta.\vy2$}
	\mauvaise{$\sin\theta.\vz2+\cos\theta.\vy2$}
	\mauvaise{$\cos\theta$}
\end{reponses}
\end{minipage}
\end{minipage} 
\end{question}
%*************************%
}
	
\element{questionsFigure1}{
%*************************%
\begin{question}{produitScalaire1}
Sur la même figure, calculer $\vy1\cdot\vz1$:
\begin{reponseshoriz}
	\bonne{$0$}
	\mauvaise{$\cos\theta$}
	\mauvaise{$-\cos\theta$}
	\mauvaise{$\sin\theta$}
	\mauvaise{$-\sin\theta.\vz1$}
	\mauvaise{$\vx1$}
\end{reponseshoriz}
\end{question}
%*************************%

%*************************%
\begin{question}{produitVectoriel1}
Sur la même figure, calculer $\vy2\vect\vy1$:
\begin{reponseshoriz}
	\bonne{$-\sin\theta.\vx1$}
	\mauvaise{$\sin\theta.\vx1$}
	\mauvaise{$\cos\theta.\vx1$}
	\mauvaise{$\cos\theta$}
	\mauvaise{$-\cos\theta.\vz1$}
	\mauvaise{$\sin\theta.\vx2$}
\end{reponseshoriz}
\end{question}
%*************************%
	
%*************************%
\begin{question}{produitVectoriel2}
Sur la même figure, calculer $\vy1\vect\vz2$:
\begin{reponseshoriz}
	\bonne{$\cos\theta.\vx1$}
	\mauvaise{$-\sin\theta.\vx1$}
	\mauvaise{$\sin\theta.\vx1$}
	\mauvaise{$\cos\theta$}
	\mauvaise{$-\cos\theta.\vz1$}
	\mauvaise{$\sin\theta.\vx2$}
\end{reponseshoriz}
\end{question}
%*************************%
}

\element{figure2}{
%*************************%
\begin{question}{projection2}

\noindent
\begin{minipage}{\linewidth}
\begin{minipage}[c]{.55\linewidth}
\setCouleursParametrage{black}{red}
\parametrageAngulaire{\theta}{\vy1}{\vz1}{\vx1}{\vy2}{\vz2}[\vx2]\qquad
\setCouleursParametrage{red}{UPSTIcustomColor1}
\parametrageAngulaire{\varphi}{\vz2}{\vx2}{\vy2}{\vz3}{\vx3}[\vy3]
\end{minipage}\hfill
\begin{minipage}[c]{0.40\linewidth}
Projeter \vx3 dans \bB1 :
\begin{reponses}
	\bonne{$\cos\varphi.\vx1+\sin\varphi\sin\theta.\vy1-\sin\varphi\cos\theta.\vz1$}
	\mauvaise{$\cos\varphi.\vx1-\sin\varphi\sin\theta.\vy1-\sin\varphi\cos\theta.\vz1$}
	\mauvaise{$\cos\varphi.\vx1+\sin\varphi\vy1$}
	\mauvaise{$\cos\varphi.\vx1-\sin\varphi\sin\theta.\vy1+\sin\varphi\cos\theta.\vz1$}
\end{reponses}
\end{minipage}
\end{minipage} 
\end{question}
%*************************%
}

\element{questionsFigure2}{
%*************************%
\begin{question}{produitScalaire2}
Sur la même figure, calculer $\vz3\cdot\vz1$:
\begin{reponseshoriz}
	\bonne{$\cos\varphi\cos\theta$}
	\mauvaise{$\cos\varphi$}
	\mauvaise{$-\cos\varphi\cos\theta$}
	\mauvaise{$\sin\varphi\cos\theta$}
	\mauvaise{$0$}
	\mauvaise{$\vx1$}
\end{reponseshoriz}
\end{question}
%*************************%

%*************************%
\begin{question}{produitVectoriel3}
Sur la même figure, calculer $\vy2\vect\vx3$:
\begin{reponseshoriz}
	\bonne{$-\vz3$}
	\mauvaise{$\cos\varphi.\vx2-\sin\varphi.\vz2$}
	\mauvaise{$0$}
	\mauvaise{$\cos\varphi\cos\theta.\vy1$}
	\mauvaise{$\cos\varphi.\vx2+\sin\varphi.\vz2$}
\end{reponseshoriz}
\end{question}
%*************************%
	
%*************************%
\begin{question}{derivationVectorielle1}
Sur la même figure, calculer $\derivV{\vy2}{\bB1}$:
\begin{reponseshoriz}
	\bonne{$\thetap.\vz2$}
	\mauvaise{$-\thetap.\vz2$}
	\mauvaise{$\vz2$}
	\mauvaise{$\cos\theta.\vz1$}
	\mauvaise{$-\cos\theta.\vz1$}
	\mauvaise{$0$}
\end{reponseshoriz}
\end{question}
%*************************%

%*************************%
\begin{question}{derivationVectorielle2}
Sur la même figure, calculer $\derivV{\vz3}{\bB1}$:
\begin{reponseshoriz}
	\bonne{$\varphip.\vx3-\thetap\cos\varphi.\vy3$}
	\mauvaise{$\varphip.\vx3+\thetap\sin\theta.\vy3$}
	\mauvaise{$\varphip.\vx3$}
	\mauvaise{$\vNul$}
	\mauvaise{$\varphip(\vx1+\vy1)$}
\end{reponseshoriz}
\end{question}
%*************************%

}

\element{liaisons}{
%*************************%
\begin{questionmult}{tableauLiaison}
\AMCnoCompleteMulti
\begin{reponseshoriz}[o]
\begin{minipage}{\linewidth}
\begin{center}
\begin{tabular}{c||c|c|c}
& \tCan{M}{\tCinematiqueCan{M}{2}{1}{0}{0}{1}{1}{1}{0}}{\bB{}}	& \tCan{A}{\tCinematiqueCan{A}{2}{1}{1}{1}{1}{1}{0}{0}}{\bB{}}	& \tCan{A}{\tCinematiqueCan{A}{2}{1}{1}{1}{1}{0}{0}{0}}{\bB{}}
\\\hline\hline
Sphère/Cylindre ? &\mauvaise{}		&\bonne{}		& \mauvaise{}
\\\hline
Sphérique ?	&\mauvaise{}		&\mauvaise{}		& \bonne{}
\\\hline
Appui plan ? &\bonne{}		&\mauvaise{}		& \mauvaise{}
\end{tabular}
\end{center}
\end{minipage}
\end{reponseshoriz}
\end{questionmult}
%*************************%

%*************************%
\begin{question}{trajectoires}
Parmi ces 4 propositions, une seule peut conclure la phrase suivante de manière cohérente. Laquelle ?: \og La trajectoire de $C$ appartenant à \np{5} par rapport à \np{3} est...\fg{}:
\begin{multicols}{2}
\begin{reponses}
	\bonne{...un cercle de centre $H$ et de rayon \segment{HC}}
	\mauvaise{...une rotation d'axe $\couple{H}{\vz1}$}
	\mauvaise{...une pivot d'axe $\couple{H}{\vz1}$}
	\mauvaise{...une translation}
\end{reponses}
\end{multicols}
\end{question}
%*************************%

%*************************%
\begin{question}{liaisons2}
Le torseur cinématique d'une liaison linéaire rectiligne de droite de contact \axe{A}{\vx{}} et de normale au plan \vz{} est:
\begin{multicols}{2}
\begin{reponses}
	\bonne{\tCan{A}{\tCinematiqueCan{A}{2}{1}{1}{0}{1}{1}{1}{0}}{\bB{}}}
	\mauvaise{\tCan{A}{\tCinematiqueCan{A}{2}{1}{0}{1}{1}{1}{0}{0}}{\bB{}}}
	\mauvaise{\tCan{A}{\tCinematiqueCan{A}{2}{1}{0}{1}{1}{1}{1}{0}}{\bB{}}}
	\mauvaise{\tCan{A}{\tCinematiqueCan{A}{2}{1}{1}{0}{1}{0}{1}{0}}{\bB{}}}
\end{reponses}
\end{multicols}
\end{question}
%*************************%

%*************************%
\begin{questionmult}{torseurCinematique}
Dans un torseur cinématique, la résultante est:
\begin{multicols}{2}
\begin{reponses}
	\bonne{un vecteur}
	\bonne{la colonne de gauche du torseur}
	\mauvaise{la deuxième ligne du torseur}
	\bonne{une vitesse de rotation}
	\mauvaise{une accélération}
	\mauvaise{dépendante du point considéré}
	\mauvaise{un scalaire}
\end{reponses}
\end{multicols}
\end{questionmult}
%*************************%
}

%---------------------------------%
% Début du questionnaire
%---------------------------------%
% Nombre d'exemplaires (c'est plus flexible de définir le nombre d'exemplaire directement dans AMC)
\exemplaire{1}{

% En-tête	
\UPSTIbuildPage
\UPSTIamcZoneIdentification

\vspace{-2.5em}
\section{Calcul vectoriel}
\cleargroup{calcul}
\copygroup{figure1}{calcul}
\melangegroupe{questionsFigure1}\copygroup[1]{questionsFigure1}{calcul}
\copygroup{figure2}{calcul}
\melangegroupe{questionsFigure2}\copygroup[1]{questionsFigure2}{calcul}
\restituegroupe{calcul}

\section{Cinématique}
\cleargroup{liaisons}
\melangegroupe{liaisons}\copygroup[1]{liaisons}{tout}
\restituegroupe{liaisons}

\clearpage    
	
}
\end{document}

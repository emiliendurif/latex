%ENTETE / INCLUDES
%%%%%%%%%%%%%%%%%%%%%%%%%%%%%%%%%%%%%
\documentclass[a4paper,10pt]{article}

%Includes
%=============================

%\usepackage[latin1]{inputenc} 		%Caracteres francais
\usepackage[utf8]{inputenc} 		%Caracteres francais
\usepackage[T1]{fontenc} 		%Caracteres francais
\usepackage[francais]{babel}		%On ecrit en francais
\usepackage{graphicx}			%Pour les images
\usepackage[top=2.5cm, bottom=2.5cm, left=2cm, right=2cm]{geometry}%Marges etc...
\usepackage{fancybox}			%Des boites
\usepackage[tikz]{bclogo}		%Pour faire des boites avec logo (remarques, etc...)
\usepackage{xcolor}			%Pour d�finir les couleurs
\usepackage{titlesec} 			%Pour modifier l'aspet des titres (sections, etc...)
\usepackage{soul} 			%Pour souligner
\usepackage{ifthen}			%Package des conditions
\usepackage{fancyhdr}			%Pour l'entete et le pied de page
\usepackage{lastpage}			%Pour connaitre le nombre total de page pour le pied de page
\usepackage{subfig}			%Pour faire des figures dans les figures
\usepackage{amsmath}			%Pour faire des maths (environnement ``align'' notamment)
\usepackage{amsfonts}			%Pour faire des maths (ensemble des reels, notamment)
\usepackage{amssymb}
\usepackage{mathrsfs}			%Pour faire des maths (notamment le L de la transform�e de Laplace)
\usepackage{answers}
\usepackage{pifont}			%Pour faire des chiffres entour�s
\usepackage[colorlinks=true,linktocpage]{hyperref}	%Pour faire des liens
\usepackage{calc}			%Pour faire heightof et des calculs de coordonn�es
\usepackage{import}			%Permet de g�rer les chemins relatifs des fichiers dans Latex
\usepackage{multirow}			%Pour faire des tableau avec plusieurs lignes
\usepackage{tikz}			%Pour faire des dessins
\usetikzlibrary{calc}
\usepackage{pgfplots}			%Pour faire des dessins (bis) (voir --> http://bertrandmasson.free.fr/index.php?article28/comment-faire-de-beaux-graphiques-avec-tikz-et-pgfplots)
\usepackage{enumerate}			%Pour personnaliser les listes ``enumerate''
\usepackage{wasysym}			%Pour faire le symbole diametre
\usepackage{array}			%Tableaux (notamment alignements)
\usepackage{placeins} 	%Pour utiliser la commande \FloatBarrier pour vider la m�moire des figures.
\usepackage{calc} %Pour faire des calculs
\usepackage{eurosym} %package euro
\usepackage{epstopdf} % gestion du format eps
\usepackage{hyperref}%Gestion des liens hypertextes
\usepackage{fancyvrb}%Gestion des boites pour l'informatique
%\usepackage{listings}
\usepackage{listingsutf8}
\usepackage{fourier}
\lstset{
	language={Python},
	columns=flexible,
	basicstyle=\ttfamily,
	keywordstyle=\color{blue}, % je voudrais mettre les mots-clé en gras (\bfseries) mais c'est incompatible avec typewriter
	commentstyle=\color{gray}, % commentaires
	backgroundcolor=\color{green!10}, % couleur du fond
	frame=single, rulecolor=\color{green!10}, % encadrement + couleur de l'encadrement
	%numbers=left, numberstyle=\tiny, stepnumber=1, numbersep=5pt,
	showstringspaces=false, % supprime les espaces apparents dans les lignes de texte
	stringstyle=\color{red!50}\itshape,
	inputencoding=utf8,
	extendedchars=true,
   literate=
  {á}{{\'a}}1 {é}{{\'e}}1 {í}{{\'i}}1 {ó}{{\'o}}1 {ú}{{\'u}}1
  {Á}{{\'A}}1 {É}{{\'E}}1 {Í}{{\'I}}1 {Ó}{{\'O}}1 {Ú}{{\'U}}1
  {à}{{\`a}}1 {è}{{\`e}}1 {ì}{{\`i}}1 {ò}{{\`o}}1 {ù}{{\`u}}1
  {À}{{\`A}}1 {È}{{\'E}}1 {Ì}{{\`I}}1 {Ò}{{\`O}}1 {Ù}{{\`U}}1
  {ä}{{\"a}}1 {ë}{{\"e}}1 {ï}{{\"i}}1 {ö}{{\"o}}1 {ü}{{\"u}}1
  {Ä}{{\"A}}1 {Ë}{{\"E}}1 {Ï}{{\"I}}1 {Ö}{{\"O}}1 {Ü}{{\"U}}1
  {â}{{\^a}}1 {ê}{{\^e}}1 {î}{{\^i}}1 {ô}{{\^o}}1 {û}{{\^u}}1
  {Â}{{\^A}}1 {Ê}{{\^E}}1 {Î}{{\^I}}1 {Ô}{{\^O}}1 {Û}{{\^U}}1
  {œ}{{\oe}}1 {Œ}{{\OE}}1 {æ}{{\ae}}1 {Æ}{{\AE}}1 {ß}{{\ss}}1
  {ű}{{\H{u}}}1 {Ű}{{\H{U}}}1 {ő}{{\H{o}}}1 {Ő}{{\H{O}}}1
  {ç}{{\c c}}1 {Ç}{{\c C}}1 {ø}{{\o}}1 {å}{{\r a}}1 {Å}{{\r A}}1
  {€}{{\EUR}}1 {£}{{\pounds}}1
}
%\usepackage{minted}
\usepackage{multicol}
\usepackage{algorithm2e}
\usepackage{pgf, tikz} % dessins et courbes
%\usepackage{subcaption}
%\usepackage{tikz,pgfplots}
%\usepackage{tkz-tab} % tableaux de variations
%\usetikzlibrary{calc}
%%\usepackage{pgf}
%\usepackage{pgfplots}
%\usetikzlibrary{babel}
%\usepackage{circuitikz} % l est un "L" minuscule
\DeclareSymbolFontAlphabet{\mathcal}{symbols}
\usepackage{color}
\usepackage{colortbl}
\newcommand*\lstinputpath[1]{\lstset{inputpath=#1}}

\definecolor{gris25}{gray}{0.75}
\definecolor{bleu}{RGB}{50,50,220}
\definecolor{bleuf}{RGB}{42,94,171}
\definecolor{bleuc}{RGB}{231,239,247}
\definecolor{rougef}{RGB}{185,18,27}
\definecolor{rouge}{RGB}{255,40,40}
\definecolor{rougec}{RGB}{255,230,231}
\definecolor{vertf}{RGB}{103,126,82}
\definecolor{vert}{RGB}{40,150,40}
\definecolor{vertc}{RGB}{220,255,191}
\definecolor{violetf}{RGB}{112,48,160}
\definecolor{violetc}{RGB}{230,224,236}
\definecolor{jaunec}{RGB}{220,255,191}
\definecolor{jaune}{RGB}{255,246,100}
%\usepackage{rpcinematik}

%\usepackage[table]{xcolor}%Mettre des coleurs dans les tableaux

\usepackage{float}
\usepackage{bodegraph}

\usepackage{booktabs}
\usepackage{rotating}

\usepackage{amsmath}
\usepackage[overload]{empheq}

\usepackage{longtable,booktabs}

\usepackage{siunitx}

\usepackage{textcomp}

%\usepackage{stmaryrd}
\usepackage{pythontex}
\usepackage{supertabular}

\usepackage{esvect}

\usepackage[breakable]{tcolorbox}

\usepackage{standalone}
\usepackage{schemabloc}
\usepackage{cancel}

%%% Mise en forme des références croisées
\newcommand{\citedoc}[1]{\textbf{document \ref{#1}}}
\newcommand{\citequest}[1]{\textbf{question Q\ref{#1}}}
\newcommand{\citefig}[1]{\textbf{figure \ref{#1}}}
\newcommand{\citetab}[1]{\textbf{tableau~\ref{#1}}}
\newcommand{\citeannexe}[1]{\textbf{annexe \ref{#1}}}

%% Gestion des numéros de parties, paragraphes et questions
\newcounter{num_probleme} \setcounter{num_probleme}{0}
\newcounter{num_exercice} \setcounter{num_exercice}{0}
\newcounter{num_quest} \setcounter{num_quest}{0}
\newcounter{num_partie} \setcounter{num_partie}{0}
\newcounter{num_sspartie} \setcounter{num_sspartie}{0}
\newcounter{num_doc} \setcounter{num_doc}{0}
\newcounter{num_annexe} \setcounter{num_annexe}{0}


\newcommand{\probleme}[1]{\stepcounter{num_probleme}
			\setcounter{num_partie}{0}\vspace{2mm}
			{\begin{center} \textbf{\large PROBL\`{E}ME
			\arabic{num_probleme}\\ \vspace{2mm}#1}\end{center}
			\vspace{2mm}}}
%\newcommand{\titre}[1]{\begin{center} \textbf{\large #1}\end{center}
%			\vspace{2mm}}
%\newcommand{\exercice}[1]{\stepcounter{num_exercice}
%			\setcounter{num_partie}{0}\vspace{2mm}
%			{\begin{center} \textbf{\large EXERCICE
%			\arabic{num_exercice}\\ \vspace{2mm}#1}\end{center}
%			\vspace{2mm}}}
\newcommand{\paragraphe}[1]{\vspace{4mm} {\textbf{#1}
			\vspace{2mm}}}
\newcommand{\partie}[1]{\stepcounter{num_partie}\setcounter{num_sspartie}{0}
		   {\vspace{2mm}\begin{center} \textbf{\begin{large}Partie
		    \Roman{num_partie} - \end{large}
		    \begin{large}#1\end{large}}\\ \par \end{center}}}
\newcommand{\souspartie}[1]{\stepcounter{num_sspartie}{\vspace{2mm} {
		    \textbf{\Roman{num_partie}.\arabic{num_sspartie} - #1}
			\vspace{2mm}}}}



\usepackage{rp-sysml}
\usepackage{rpcinematik}
\usepackage{schemabloc}

\usetikzlibrary{calc}
\usepgflibrary{arrows}
\usetikzlibrary{trees}
\usetikzlibrary{plotmarks}

\usepackage{pdfpages}%inclure un pdf complet

\usepackage{xargs}


\usepackage{siunitx}

\makeatletter
\import{../}{notations.tex}



%Mise à zéro des variables
%==================================


%\newcommand	{\partie}		{Sciences de l'ingénieur}
%\newcommand	{\titre}		{Analyse fonctionnelle}
%\newcommand	{\numero}		{1}
%\newcommand	{\auteur}		{Emilien DURIF}
%\newcommand	{\etablissement}	{Lycée Gustave Eiffel de Dijon}
%\newcommand	{\discipline}		{Sciences de L'ingénieur}
%\newcommand	{\classe}		{Classe préparatoire P.T.S.I.}
%\newcommand	{\annee}		{2011 - 2012}
%\newcommand	{\icone}		{../../latex/images/logo_eiffel.png}
%\newcommand	{\competences}		{}

%\newcommand	{\partie}		{Statique des solides}
\newcommand	{\cycle}		{C1 : Performances statiques et cinématiques des systèmes composés de chaine de solides}
\newcommand	{\cycleresume}		{C2 : Chaine de solides}
\newcommand	{\titreresume}	{}
\newcommand	{\titre}		{Modélisation des actions mécaniques et résolution d'un problème de statique}
\newcommand	{\numero}		{C2-1}
\newcommand	{\numerotd}		{1}
\newcommand	{\auteur}		{Emilien DURIF}
\newcommand	{\etablissement}	{Lycée La Martinière Monplaisir Reims}
\newcommand	{\discipline}		{Sciences Industrielles pour l'Ingénieur}
\newcommand	{\classe}		{Classe préparatoire P.S.I.}
\newcommand	{\annee}		{2016 - 2017}
\newcommand	{\icone}		{../../../../latex/images/logo_martiniere.jpg}
\newcommand	{\competences}		{}



%Mise en page des titres de parties
%======================================

%Sections
\renewcommand\thesection{\Roman{section}}%Numérotation en chiffres romains
\titleformat{\section}[hang]{\LARGE\bfseries}{\thesection.}{1em}{}[\rule{\linewidth}{.5pt}]
\newcommand{\sectionbreak}{\clearpage}%Change de page a chaque section
%\titlespacing*{\section}{0px}{20cm}{\wordsep}

%Subsection
\renewcommand\thesubsection{\arabic{subsection}}%Numérotation en chiffres arabes
\titlespacing*{\subsection}{0cm}{1cm}{0.5cm}

%Subsubsection
\renewcommand\thesubsubsection{\alph{subsubsection})}%Numérotation en lettres
\titlespacing*{\subsubsection}{1cm}{0.5cm}{0.3cm}





%Entete / Pied de page
%===========================
\pagestyle{fancy}	%On veut utiliser les entetes/pieds
\rhead{\numero\;\titreresume}		%Entete gauche --> Titre du document
\lhead{\textsc{\cycleresume}}	%Entete droite --> Titre de la sections

\renewcommand{\footrulewidth}{1px}
\lfoot{\etablissement}	%Pied de gauche --> Etablissement
\cfoot{\thepage\ / \pageref{LastPage}}	%Pied Centre --> n°de page
\rfoot{\classe \\ Année \annee}











%Booléen de texte à trou
\newboolean{texteATrou}
\setboolean{texteATrou}{false}	%Petite condition qui choisit entre 2 formats d'image
\newcommand{\setTexteATrouOn}{\setboolean{texteATrou}{true}}	%Active le texte à trous
\newcommand{\setTexteATrouOff}{\setboolean{texteATrou}{false}}	%Désactive le texte à trous
\definecolor{couleurTrou}{rgb}{0,0.5,0}
\newcommand{\trou}[1]{\ifthenelse{\boolean{texteATrou}}{\ifthenelse{\boolean{makingOf}}{\textcolor{couleurTrou}{#1}}{\hphantom{#1}}}{#1}}

%Booleen d'affichage vectoriel ou bitmap
\newboolean{imageEnVectoriel}
\setboolean{imageEnVectoriel}{true}	%Petite condition qui choisit entre 2 formats d'image
\newboolean{corrige}
\setboolean{corrige}{true}	%Petite condition qui choisit entre 2 formats d'image
%Booleen d'affichage des corrections de TD
\newboolean{tdcorrige}
\setboolean{tdcorrige}{false}




\newboolean{correction}
\setboolean{correction}{false}
\newcommand{\correction}[2][]{\ifthenelse{\boolean{correction}}{#2}{#1}}

%\newcommand{\makingOf}[1]{\ifthenelse{\boolean{makingOf}}{}{#1}}









%Mode DISPLAY STYLE dans les tableau
%%%% debut macro %%%%
\newenvironment{disarray}%
 {\everymath{\displaystyle\everymath{}}\array}%
 {\endarray}
%%%% fin macro %%%%

%Pour ne pas apparaitre dans le menu$  $
%exemple :
% \tocless\subsection{blabla}
\newcommand{\nocontentsline}[3]{}
\newcommand{\tocless}[2]{\bgroup\let\addcontentsline=\nocontentsline#1{#2}\egroup}


%ITEMIZE

\AtBeginDocument{%
	\renewcommand{\labelitemi}	{\textbullet}
	\renewcommand{\labelitemii}	{$\circ$}%
	\renewcommand{\labelitemiii}	{>}%
	\renewcommand{\labelitemiv}	{-}%
}



%%%%Commande au début du document latex
%Lettres grecques oubliees
\newcommand{\Mu}{M} %mu majuscule

%Mise en forme
%\newcommand{\gras}[1]	{\textbf{#1}}
\newcommand{\bouton}[1]	{\fbox{\footnotesize{\textsc{#1}}}}
\newcommand{\toutpetit}[1]	{{\tiny{1}}}
\newcommand{\PETIT}[1]		{{\scriptsize{#1}}}
\newcommand{\Petit}[1]		{\footnotesize{#1}}
\newcommand{\petit}[1]		{{\small{#1}}}
\newcommand{\normal}[1]		{{\normalsize{#1}}}
\newcommand{\grand}[1]		{{\large{#1}}}
\newcommand{\Grand}[1]		{{\Large{#1}}}
\newcommand{\GRAND}[1]		{{\LARGE{#1}}}
\newcommand{\enorme}[1]		{{\huge{#1}}}
\newcommand{\Enorme}[1]		{{\Huge{#1}}}

%\newcommand{\oeuvre}		{\oe uvre}
%\newcommand{\oeuvres}		{\oe uvres}

\newcommand{\fig}[1]		{(fig.\ref{#1})}

\newcommand{\miniCentre}[2][\linewidth]	{\begin{minipage}{#1}\begin{center}#2\end{center}\end{minipage}}
%\newcommand{\sectionbreak}{}





%Infos pages
%%%%%%%%%%%%%%%%%%%%%%%%%%%%%%%%%%
\renewcommand	{\partie}		{Cinématique}
\renewcommand	{\titre}		{Cinématique du solide}
\renewcommand	{\numero}		{1}
\renewcommand	{\auteur}		{Raphaël ALLAIS}
\renewcommand	{\etablissement}	{Lycée Gustave Eiffel de Dijon}
\renewcommand	{\discipline}		{Sciences de L'ingénieur}
\renewcommand	{\classe}		{Classe préparatoire P.T.S.I.}
\renewcommand	{\annee}		{2011 - 2012}
\renewcommand	{\icone}		{../../latex/images/logo_eiffel.png}


%DEBUT DU DOCUMENT
%%%%%%%%%%%%%%%%%%%%%%%%%%ù
\begin{document}
	\subsection{Commandes de base}
	%------------------------------------
	\newcommand{\bs}{\textbackslash}

		\subsubsection{Lettres grecques oubliées}
		%...........................................

			\begin{tabular}{|c|c|c|}
				\hline
					\bs Mu			&	\Mu			& Le $\mu$ grecque majuscule\\
				\hline
			\end{tabular}

		\subsubsection{Mise en forme du texte}
		%........................................
			\begin{tabular}{|c|c|c|}
				\hline
					\bs gras\{mon texte\}	&	\gras{mon texte}	& Mettre le texte en gras (identique à \bf textbf)\\
				\hline
					\bs bouton\{Mon texte\}	&	\bouton{mon texte}	& Fait un bouton (pour représenter une icône, par exemple)\\
				\hline
			\end{tabular}

	\subsection{Maths générales}
	%.................................

		\begin{tabular}{|c|c|c|}
			\hline
				\bs R					&	\R			& Ensemble des réels\\
			\hline
				\bs ssi					&	\ssi			& \\
			\hline
				\bs indiceGauche\{i\}\{X\}		&	\indiceGauche{i}{X}	& Indice à gauche \\
			\hline
				\bs exposantGauche\{i\}\{X\}		&	\exposantGauche{i}{X}	& Exposant à gauche \\
			\hline
				\bs transposee\{M\}			&	$\transposee{M}$	& Symbole transposée\\
			\hline
				\bs fonction\{f\}\{x\}			&	\fonction{f}{x}		& Met en forme une fonction\\
			\hline
				\bs f\{g\}\{x\}				&	\f{g}{x}		& Raccourci de \bs fonction\\
			\hline
				\bs deriv\{f\}\{t\}			&	\deriv{f}{t}		& Dérivation \\
			\hline
				\bs deriv[R]\{f\}\{t\}			&	\deriv[R]{f}{t}		& Dérivation dans un repère\\
			\hline			
				\bs angle\{ABC\}			&	\angle{ABC}		& Met en forme un angle (remplace le symbole angle)\\
			\hline
				\bs couple\{A\}\{B\}			&	\couple{A}{B}		& Couple d'objets\\
			\hline
				\bs triplet\{A\}\{B\}\{C\}		&	\triplet{A}{B}{C}		& Triplet d'objets\\
			\hline
				\bs quadruplet\{A\}\{B\}\{C\}\{D\}	&	\quadruplet{A}{B}{C}{D}		& Quadruplet d'objets\\
			\hline
				\bs segment\{AB\}			&	\segment{AB}				& Segment\\
			\hline
				\bs droite\{AB\}			&	\droite{AB}				& Droite\\
			\hline
		\end{tabular}



	\subsection{Analyse fonctionnelle}
	%-------------------------------------

			\begin{tabular}{|c|c|c|}
				\hline
					\bs MO				&	\MO			& \\
				\hline
					\bs VA				&	\VA			& \\
				\hline
			\end{tabular}


	\subsection{Asservissements}
	%---------------------------------

			\begin{tabular}{|c|c|c|}
				\hline
					\bs TOR				&	\TOR			& \\
				\hline
					\bs L				&	\L			& Symbole de Laplace (ancien L barré)\\
				\hline
			\end{tabular}

	\subsection{Vecteurs}
	%--------------------------------

		\subsubsection{Commandes de base}
		%...................................

			\begin{tabular}{|c|c|p{10cm}|}
				\hline
					\bs vecteur\{V\}			&	\vecteur{V}			& écriture de base d'un vecteur\\
				\hline
					\bs vecteurIndice\{V\}\{i\},	&	\vecteurIndice{V}{i}		& vecteur avec un indice. Si l'indice est un espace, alors le vecteur redevient classique\\
				\hline
					\bs vecteurChamp\{OP\}\{t\}		&	\vecteurChamp{OP}{t}		& combinaison entre \bs vecteur et \bs fonction \\
				\hline
					\bs bipoint\{A\}\{B\}		&	\bipoint{A}{B}		& Bipoint.\\
				\hline
			\end{tabular}

		\subsubsection{Espaces}
		%...........................

			\emph{(Le nom des espaces commence toujours par un ``e'')}

			\begin{tabular}{|c|c|p{10cm}|}
				\hline
					\bs eAffine		&	\eAffine		& Espace Affine de dimension 3.	\\
				\hline
					\bs eAffine[n]		&	\eAffine[n]		& Espace Affine de dimension n.	\\
				\hline
					\bs eVectoriel		&	\eVectoriel		& Espace Vectoriel de dimension 3.\\
				\hline
					\bs eVectoriel[n]	&	\eVectoriel[n]		& Espace Vectoriel de dimension n.\\
				\hline
			\end{tabular}

		\subsubsection{Représentation des vecteurs}
		%............................................;

			\begin{tabular}{|c|c|p{10cm}|}
				\hline
					\bs vColonne\{x\}\{y\}\{z\}	&	\vColonne{x}{y}{z}	& Vecteur colonne (sans la base).\\
				\hline
					\bs vColonne[B]\{x\}\{y\}\{z\}	&	\vColonne[B]{x}{y}{z}	& Vecteur colonne (avec base).\\
				\hline
			\end{tabular}

		\subsubsection{Opérateurs vectoriels}
		%.......................................

			\begin{tabular}{|c|c|p{10cm}|}
				\hline
					\bs norme\{X\}				&	\norme{X}			& Norme.\\
				\hline
					\bs abs\{X\}				&	\abs{X}				& Valeur absolue / module.\\
				\hline
					\bs prodMixte\{A\}\{B\}\{C\}		&	\prodMixte{A}{B}{C}		& Produit mixte.\\
				\hline
					\bs doubleProdVect\{A\}\{B\}\{C\}	&	\doubleProdVect{A}{B}{C}	& Double produit vectoriel\\
				\hline
			\end{tabular}


		\subsubsection{Vecteur Pré-fabriqués}
		%.......................................

			\emph{(Tous les noms de vecteur commencent par la lettre ``v'')}

			\begin{tabular}{|c|c|p{10cm}|}
				\hline
					\bs vNul				&	\vNul			& Vecteur Nul\\
				\hline
					\bs vPreFab\{U\}			&	\vPreFab{U}			& Vecteur qui permet de choisir entre un vecteur ou un champ
															(Certains vecteurs héritent de ça)	\\
				\hline
					\bs vPreFab[M]\{U\}			&	\vPreFab[M]{U}			& Vecteur qui permet de choisir entre un vecteur ou un champ
															(Certains vecteurs héritent de ça)	\\
				\hline
					\bs ve\{i\}				&	\ve{i}			& Vecteurs e...	\\
				\hline
					\bs ve1 				&	\ve1		& (valable pour tous chiffres \ve2, \ve3, ...)\\
				\hline
					\bs vex					&	\vex 			& \\
				\hline
					\bs vey					&	\vey 			& \\
				\hline
					\bs vez					&	\vez 			& \\
				\hline
					\bs vu					&	\vu 			& \\
				\hline
					\bs vu[M]				&	\vu[M] 			& \\
				\hline
					\bs vv					&	\vv 			& \\
				\hline
					\bs vv[M]				&	\vv[M] 			& \\
				\hline
					\bs vw					&	\vw 			& \\
				\hline
					\bs vw[M]				&	\vw[M] 			& \\
				\hline
					\bs vU					&	\vU 			& \\
				\hline
					\bs vU[M]				&	\vU[M] 			& \\
				\hline
					\bs vV					&	\vV 			& \\
				\hline
					\bs vV[M]				&	\vV[M] 			& \\
				\hline
					\bs vW					&	\vW 			& \\
				\hline
					\bs vW[M]				&	\vW[M] 			& \\
				\hline
					\bs vOM					&	\vOM 			& \\
				\hline
					\bs OM					&	\vOM 			& (identique à \bs vOM)\\
				\hline
					\bs vOM[M]				&	\vOM[M]			& \\
				\hline
					\bs vOP					&	\vOP			& \\
				\hline
					\bs OP					&	\vOP 			& (identique à \bs vOP)\\
				\hline
					\bs vOP[M]				&	\vOP[M]			& \\
				\hline
					\bs vi					&	\vi			& \\
				\hline
					\bs vi\{0\}				&	\vi0			& \\
				\hline
					\bs vj					&	\vj			& \\
				\hline
					\bs vj\{0\}				&	\vj0			& \\
				\hline
					\bs vk					&	\vk			& \\
				\hline
					\bs vk\{0\}				&	\vk0			& \\
				\hline
			\end{tabular}

		\subsubsection{Bases}
		%.......................;

			\begin{tabular}{|c|c|p{10cm}|}
				\hline
					\bs B					&	\B				& Symbole de base.\\
				\hline
					\bs Bxyz				&	\Bxyz				& Base x y z\\
				\hline
					\bs Buvw				&	\Buvw				& Base u v w\\
				\hline
			\end{tabular}




	\subsection{Torseurs}
	%-----------------------

		\subsubsection{Globalament}
		%..............................

			\begin{tabular}{|c|c|p{10cm}|}
				\hline
					\bs T					&	\T				& Symbole torseur.\\
				\hline
					\bs torseur\{X\}			&	\torseur{X}			& Torseur X.\\
				\hline
					\bs tT					&	\tT				& Torseur \T	\\
				\hline
			\end{tabular}

		\subsubsection{Éléments de réduction}
		%........................................;

			\begin{tabular}{|c|c|p{10cm}|}
				\hline
					\bs M					&	\M				& ``M'' de moment\\
				\hline
					\bs resultante				&	\resultante			& Résultante (générique) du torseur (\torseur{\T} par défaut)\\
				\hline
					\bs resultante[1/2]			&	\resultante[1/2]		& Résultante (générique)\\
				\hline
					\bs moment\{A\}				&	\moment{A}			& Moment au point A du torseur (\torseur{\T} par défaut)\\
				\hline
					\bs moment[1/2]\{A\}			&	\moment[1/2]{A}		& Moment au point A\\
				\hline
			\end{tabular}

			\begin{tabular}{|c|c|l|}
				\hline
					\bs torseurLigne\{A\}\{\bs vu\}\{\bs vv\}	&	\torseurLigne{A}{\vu}{\vv}			& torseur ligne\\
				\hline
					\bs tLigne\{A\}\{\bs vu\}\{\bs vv\}		&	\tLigne{A}{\vu}{\vv}				& raccourci de \bs torseurLigne\\
				\hline
					\bs torseurColonne\{A\}\{X\bs\bs Y\bs\bs Z\}\{L\bs\bs M\bs\bs N\}\{R\}	&	\torseurColonne{A}{X\\Y\\Z}{L\\M\\N}{R}	& Torseur colonne\\
				\hline
					\bs tColonne\{A\}\{X\bs\bs Y\bs\bs Z\}\{L\bs\bs M\bs\bs N\}\{R\}	&	\tColonne{A}{X\\Y\\Z}{L\\M\\N}{R}	& Raccourci de \bs torseurColonne\\
				\hline
			\end{tabular}

		\subsubsection{Opérateurs}
		%............................;

			\begin{tabular}{|c|c|l|}
				\hline
					\bs automoment 		&	\automoment		& Automoment (par défaut de \tT)\\
				\hline
					\bs automoment[1/2]	&	\automoment[1/2]	& Automoment\\
				\hline
			\end{tabular}




	\subsection{cinématique}
	%------------------------------


			\begin{tabular}{|c|c|l|}
				\hline
					\bs V				&	\V		& Symbole du torseur cinématique\\
				\hline
					\bs tCinematique\{1\}\{2\}	&	\tCinematique{1}{2}		& Torseur cinématique\\
				\hline
					\bs tV\{1\}\{2\}		&	\tV{1}{2}		& Raccourci de \bs tCinematique \\
				\hline
			\end{tabular}







	\subsection{Les boîtes}
	%------------------------




		\subsubsection{Définitions :}

			\fbox{	\begin{minipage}{10cm}
					\gras{\bs begin\{}definition\gras{\}}[Titre optionnel]\\
						Ma définition\\
					\gras{\bs end\{}definition\gras{\}}
				\end{minipage}}

			\begin{definition}[Titre optionnel]
				Ma définition
			\end{definition}



			\fbox{	\begin{minipage}{10cm}
					\gras{\bs begin\{}definitions\gras{\}}[Titre optionnel]\\
						\gras{\bs item} définition 1\\
						\gras{\bs item} définition 2\\
					\gras{\bs end\{}definitions\gras{\}}
				\end{minipage}}

			\begin{definitions}[Titre optionnel]
				\item définition 1
				\item définition 2
			\end{definitions}





		\subsubsection{Remarques :}

			\fbox{	\begin{minipage}{10cm}
					\gras{\bs begin\{}remarque\gras{\}}[Titre optionnel]\\
						\qquad Ma remarque\\
					\gras{\bs end\{}remarque\gras{\}}
				\end{minipage}}

			\begin{remarque}[Titre optionnel]
				Ma remarque
			\end{remarque}




		\fbox{	\begin{minipage}{10cm}
				\gras{\bs begin\{}remarques\gras{\}}[Titre optionnel]\\
					\gras{\bs item} première remarque\\
					\gras{\bs item} deuxième remarque\\
				\gras{\bs end\{}remarques\gras{\}}
			\end{minipage}}

		\begin{remarques}[Titre optionnel]
			\item première remarque
			\item deuxième remarque
		\end{remarques}





		\subsubsection{Attention :}

			\fbox{	\begin{minipage}{10cm}
					\gras{\bs begin\{}attention\gras{\}}[Titre optionnel]\\
						Il faut faire attention !\\
					\gras{\bs end\{}attention\gras{\}}
				\end{minipage}}



			\begin{attention}[Titre optionnel]
				Il faut faire attention
			\end{attention}




		\subsubsection{Important :}

			\fbox{	\begin{minipage}{10cm}
					\gras{\bs begin\{}important\gras{\}}[Titre optionnel]\\
						Chose importante.\\
					\gras{\bs end\{}important\gras{\}}
				\end{minipage}}



			\begin{important}[Titre optionnel]
				Chose importante.
			\end{important}


\end{document}
